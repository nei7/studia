\documentclass[12pt,twoside]{book}

\usepackage[polish]{babel}
\usepackage[utf8]{inputenc}
\usepackage[T1]{fontenc}
\usepackage{graphicx}
\usepackage{float}
\usepackage{lipsum}
\usepackage{tikz}
\usetikzlibrary{positioning}

\author{Franciszek Szarek}
\title{Skład tekstu w praktyce}
\date{}

\begin{document}
\tableofcontents   
\listoffigures      
\listoftables        

\newpage 

\chapter{Wprowadzenie}
\section{Cel książki}
\lipsum[1-2]

\section{Zakres tematyczny}
\lipsum[3-4]

\section{Krój pisma}

\textbf{ To jest tekst pogrubiony. } \\
\textit{To jest tekst pochylony.} \\ 
\textsl{To jest tekst napisany kursywą.} \\
\underline{To jest tekst podkreślony.} \\
\verb!To jest tekst zapisany jako maszynopis.! \\
{ \huge To jest tekst większą czcionką.} \\
\small{To jest tekst mniejszą czcionką.} \\


\chapter{Srodowisko}
\section{Zdefiniować wyliczenie}
\begin{enumerate}
\item pierwszy krok
\item drugi krok
\item trzeci krok
\end{enumerate}

\section{Wypunktowanie}
\begin{itemize}
\item opcja A
\item opcja B
\end{itemize}

\section{Zdefiniować środowsko rysunku wyrównanego}
\begin{figure}[h!]
    \includegraphics[width=0.5\textwidth]{obrazek.jpg}
\end{figure}

\section{Tabela}
\begin{table}[H]
    \centering
    \caption{Przykładowa tabela 3x5}
    \label{tab:przykladowa_tabela}
    \begin{tabular}{c c c}
        Nagłówek 1 & Nagłówek 2 & Nagłówek 3 \\
        \hline
        A & B & C \\
        D & E & F \\
        G & H & I \\
        J & K & L \\
        M & N & O \\
    \end{tabular}
\end{table}

\section{}
\begin{table}[h]
    \centering
    \caption{Sinus kata}
    \label{tab:dsfs}
    \begin{tabular}{c c c}
        naglowek & naglowek & naglowek \\
        \hline
        kat & 12 & 1231 \\ 
        sin & 0.5 & 0.866 \\ 
    \end{tabular}
\end{table}


\chapter{Wzory matematyczne}
\section{Wzór na pole koła}
\begin{equation}
   P = \pi r^2
\label{eq:pole_kola}
\end{equation}

\section{Wzór na wyróżnik trójmianu kwadratowego oraz jego miejsca zerowe}
\begin{equation}
    \Delta = b^2 - 4ac
    \label{eq:delta}
\end{equation}

\begin{equation}
    x_1 = \frac{-b - \sqrt{\Delta}}{2a}
    \label{eq:x1}
\end{equation}

\begin{equation}
    x_2 = \frac{-b + \sqrt{\Delta}}{2a}
    \label{eq:x2}
\end{equation}

\chapter{Odwołania}
\section{Zdania z odwołaniami do etykiet}
Rysunek \ref{fig:przykladowy_rysunek},
Tabela \ref{tab:przykladowa_tabela} Pole koła \ref{eq:pole_kola}


\section{Przypis dolny}
Przypis dolny\footnote{Przypis dolny}. adsfsd

\section{Znaki specjalne}
\$, \#, \%, \^{ }, \&, \"{ }, \[ \], \textbackslash, \~{ }. \{ \}. \textbar{}. \_ @,

\chapter{Rysunki w PGF/TikZ}
\section{}
\begin{figure}[!htp]
\centering
\begin{tikzpicture}[node distance=1.25cm,main/.style = {draw, circle}]
\node[main] (1) {a};
\node[main] (2) [below right of=1] {b};
\node[main] (3) [above right of=1] {c};
\node[main,draw=none] (4) [right of=3] {~};
\draw[->] (1) to (2);
\draw[->,color=green] (1) to (3);
\draw[->,dashed,color=blue] (3) to (4);
\end{tikzpicture}
\end{figure}

\section{}
\begin{figure}[h]
    \centering
    \begin{tikzpicture}

        \node[draw, circle] (a) at (0,0) {a};
        \node[draw, circle] (c) at (1.5, 1) {c};
        \node[draw, circle] (b) at (1.5, -1) {b};



        \node[draw, circle] (delta) at (5, 0) {$\Delta$};
        \node[draw, circle] (z) at (7, -1) {z};


        \draw[->, color=green] (a) -- (c);     
        \draw[->] (a) -- (b);                 
        
        \draw[->] (c) -- (delta); 


        \draw[->, dotted] (a) -- (delta);
        
        \draw[->] (b) -- (delta);
        
        \draw[->, color=red] (b) -- (z);
        
        \draw[->, color=red] (z) -- (delta);
        \draw[->, dashed, color=blue] (c) -- +(1.0, 0);

    \end{tikzpicture}
    \caption{Końcowy graf do samodzielnego utworzenia}
\end{figure}



\end{document}
